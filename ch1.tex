\documentclass{article}
\usepackage{amsmath, amssymb, amsthm, graphicx}
\usepackage[export]{adjustbox}

\title{Chapter 1}
\author{Andrew Taylor}
\date{May 9 2022}
\newtheorem{theorem}{Theorem}
\newtheorem{problem}{Problem}
\newtheorem*{solution}{Solution}
\newtheorem{example}{Example}

\begin{document}
\maketitle

\begin{example}[Division by zero] 
The cancellation law tells us that $ac = bc$ implies $a = b$ except when $c = 0$. For example, $1 \times 0 = 2 \times 0$ does not imply that $1 = 2$, because we cannot divide by zero. This teaches us that in math there is sometimes an exception to a rule. When it comes to the cancellation law, division by zero is the exception to the rule; we cannot divide by zero. So $c = 0$ is an exception to the cancellation law, and it is also the only exception to the cancellation law. When $c \neq 0$, we can say that $a = b$.
\end{example}

\end{document}